para el invariante de rep de iterador:
1)  esRedBlack(header) = = TRUE
2) hayHader(it.Nodo) == TRUE  ????? (b�sicamente un auxiliar booleano que dice que con una cantidad finita de iteraciones NODO->parent se alcanza el reader)
3) EsPermutacion(SecuSuby(it.nodo), abToSecu()) ??Abs???

Auxiliares explicados en castellano:

insertFixUp
La funci�n de inserci�n acude a este auxiliar para arreglar posibles violaciones del invariante de representaci�n de la estructura del �rbol Red-Black al insertar un nuevo nodo. El par�metro de entrada es un puntero al nodo insertado por la funci�n insertar. La funci�n tiene un ciclo que abarca tres casos. En el primer caso si el padre del nodo pasado como par�metro es rojo y su t�o tambi�n entonces se invierten los colores del padre, el abuelo y el t�o, y el nodo con el que se itera pasa a ser el abuelo.El segundo caso siempre lleva al caso tres. En el segundo caso si el nodo con el que se itera  es hijo derecho entonces el nodo pasa a ser su padre y se hace una rotaci�n. Si el nodo es hijo izquierdo entonces es una rotaci�n derecha sino es rotaci�n izquierda(rotaci�n es explica en Rotate). En el tercer caso se invierten los colores del padre y el abuelo y si en el caso dos el padre es hijo derecho entonces se hace una rotaci�n derecha sino una rotaci�n izquierda. Mientras que el padre de nodo sea rojo el ciclo sigue iterando. 



Rotate
La funci�n insertFixUp llama a esta funci�n para hacer rotaciones del nodo pasado como par�metro con fines de restaurar el invariante de representaci�n del �rbol Red-Black despu�s de una inserci�n. Esta funci�n tambi�n toma como par�metro un entero que tiene que ser 1 o 0. Si este numero es 0 entonces es una rotacion derecha, es decir el nodo pasada como par�metro pasa a ser el hijo derecho del nodo que sol�a ser su hijo izquierdo y el hijo derecho del hijo izquierdo del nodo pasado como par�metro pasa a ser hijo  izquierdo del nodo pasado como par�metro. Si el numero es 1, entonces la descripci�n es la misma pero invirtiendo la palabra izquierda por derecha y derecha por izquierda.

transplant
Los parametro de esta funci�n son dos punteros a nodos, ?viejo? y ?nuevo?, que tienen que ser no nulos. Esta funci�n hace que el padre del nodo ?viejo? pase a ser el padre del nodo ?nuevo? y que el padre del nodo ?viejo? pase a tener de hijo al nodo ?nuevo?. En el caso de que el ?viejo? sea la ra�z del �rbol el header pasa a tener al ?nuevo? como padre y el ?nuevo? pasa a tener como padre al header. 

is_black
Funci�n booleana que devuelve true si y solo si el nodo pasado como pareamtro tiene color negro o si es nullpointer. El par�metro de entrada no tiene precondiciones.




minimo: conj(key) c \TIMES Key k \TO key  {k \IN c}

minimo(c,k) \EQUIV \IF \EMPTYSET(c) \THEN k \ELSE \IF  k \LT dameUno(c) \THEN minimo(sinUno(c),k)
\ELSE minimo(sinUno(c) ,dameUno(c))

menorLexico: conj(key) \TIMES conj(key) \TO bool

menorLexico(c1 ,c2) \EQUIV \IF \EMPTYSET(c2) \THEN false \ELSE \IF \EMPTYSET(c1) \THEN true \ELSE \IF minimo(c1,dameUno(c1)) = minimo (c2, dameUno(c2)) \THEN menorLexicografico(c1 \MINUS {minimo(c1,dameUno(c1))}, c2 \MINUS {minimo (c2, dameUno(c2))} )  \ELSE minimo(c1,dameUno(c1)) \LT minimo(c2,dameUno(c2))





