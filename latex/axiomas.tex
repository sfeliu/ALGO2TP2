En esta pagina, y por conveniencia, se listan todos los axiomas y proposiciones auxiliares requeridos para formalizar los invariantes de representación y las funciones de abstracción. Previamente se presentan los renombres de los tipos usados.

\begin{DoxyParagraph}{Renombres de tipos}

\end{DoxyParagraph}

\begin{DoxyItemize}
\item Node es tupla(child\+: arreglo\mbox{[}2\mbox{]} de puntero(\+Node), parent\+: puntero(\+Node), color\+: Color, value\+: Value)
\item Value es Maybe(value\+\_\+type)
\item value\+\_\+type es tupla(clave\+: Key, significado\+: Meaning)
\end{DoxyItemize}

El T\+AD Maybe( $\alpha$) representa un tipo $\alpha$ extendido con un valor nulo. En otras palabras, el T\+AD Maybe se puede usar para representar los valores de los nodos, donde el nodo cabecera no tiene valor y los nodos internos sí tienen valor. (Esto independientemente de si se implementa con herencia o con un puntero o de otra forma.) Tiene dos observadores\+:
\begin{DoxyItemize}
\item nothing?(x)\+: que indica si x tiene un valor nulo, y
\item data(x)\+: que devuelve el valor de x, suponiendo que no es inválido.
\end{DoxyItemize}

La especificación de este T\+AD queda como ejercicio (no obligatorio).\hypertarget{axiomas_sec-Axiomas}{}\subsection{Axiomas y proposiciones auxiliares}\label{axiomas_sec-Axiomas}
En esta sección se deben incluir todos los axiomas y proposiciones auxiliares que se usen para describir los invariantes de representación, las funciones de abstracción, las precondiciones y las postcondiciones.

\begin{DoxyRemark}{Comentarios}
Recordar incluir un alias en el archivo doxyfile a fin de poder referenciar automaticamente a cada axioma desde las otras páginas.
\end{DoxyRemark}
Se muestran algunos ejemplos a continuación.

\begin{DoxyParagraph}{menor\+Lexico}
compara lexicograficamente 2 conjuntos.

menor\+Lexico\+: conj(key) $\times$ conj(key) $\to$ bool menor\+Lexico(c1 ,c2) $\equiv$ {\bfseries if} $\emptyset$(c2) {\bfseries then} false {\bfseries else} {\bfseries if} $\emptyset$(c1) {\bfseries then} true {\bfseries else} {\bfseries if} minimo(c1,dame\+Uno(c1)) = minimo (c2, dame\+Uno(c2)) {\bfseries then} menor\+Lexicografico(c1 $-$ \{minimo(c1,dame\+Uno(c1))\}, c2 $-$ \{minimo (c2, dame\+Uno(c2))\} ) {\bfseries else} minimo(c1,dame\+Uno(c1)) $<$ minimo(c2,dame\+Uno(c2)) 
\end{DoxyParagraph}


\begin{DoxyParagraph}{minimo}
devuelve el elemento menor de un conjunto.

minimo\+: conj(key) c $\times$ Key k $\to$ key \{k $\in$ c\} minimo(c,k) $\equiv$ {\bfseries if} $\emptyset$(c) {\bfseries then} k {\bfseries else} {\bfseries if} k $<$ dame\+Uno(c) {\bfseries then} minimo(sin\+Uno(c),k) {\bfseries else} minimo(sin\+Uno(c) ,dame\+Uno(c)) 
\end{DoxyParagraph}


\begin{DoxyParagraph}{cantidad\+De\+Elementos}
devuelve la cantidad de elementos que tiene el diccionario

cantidad\+De\+Elementos\+: puntero(\+Node) $\to$ nat~\newline
 cantidad\+De\+Elementos(p) $\equiv$ {\bfseries if} p = null {\bfseries then} 0 {\bfseries else} {\bfseries if} nothing?(p-\/$>$value) {\bfseries then} cantidad\+De\+Elementos(p-\/$>$parent) ~\newline
 {\bfseries else} 1 + \href{axiomas.html#cantidadDeElementos}{\tt cantidad\+De\+Elementos}(p-\/$>$child\mbox{[}0\mbox{]}) + \href{axiomas.html#cantidadDeElementos}{\tt cantidad\+De\+Elementos}(p-\/$>$child\mbox{[}1\mbox{]}) 
\end{DoxyParagraph}


\begin{DoxyParagraph}{es\+A\+DB}
devuelve una bool indicando si el arbol tiene una relacion de orden total (cada nodo con sus hijos)

es\+A\+DB\+: puntero(\+Node) $\to$ bool~\newline
 es\+A\+D\+B(p) $\equiv$ {\bfseries if} p = null {\bfseries then} true {\bfseries else} {\bfseries if} nothing?(p-\/$>$value) {\bfseries then} es\+A\+DB(p-\/$>$parent) {\bfseries else} {\bfseries if} (¬(p-\/$>$child\mbox{[}0\mbox{]}=null) $\land_{\rm L}$ $\pi_1$ (p-\/$>$child\mbox{[}0\mbox{]}-\/$>$value)) $\lor_{\rm L}$ (¬(p-\/$>$child\mbox{[}1\mbox{]}=null) $\land_{\rm L}$ $\pi_1$ (p-\/$>$child\mbox{[}1\mbox{]}-\/$>$value)) {\bfseries then} true ~\newline
 {\bfseries else} \href{axiomas.html#esADB}{\tt es\+A\+DB}(p-\/$>$child\mbox{[}0\mbox{]}) $\land_{\rm L}$ \href{axiomas.html#esADB}{\tt es\+A\+DB}(p-\/$>$child\mbox{[}1\mbox{]}) 
\end{DoxyParagraph}


\begin{DoxyParagraph}{cant\+Black}
dado un puntero nod, devuelve la cantidad de nodos negros hay hasta llegar al root

cant\+Black\+: puntero(\+Node) $\to$ nat~\newline
 cant\+Black(p) $\equiv$ {\bfseries if} p = null {\bfseries then} 0 {\bfseries else} {\bfseries if} nothing?(p-\/$>$value) {\bfseries then} 0 {\bfseries else} {\bfseries if} p = p-\/$>$parent-\/$>$parent {\bfseries then} 1 {\bfseries else} {\bfseries if} p-\/$>$color = black {\bfseries then} 1 + (p-\/$>$parent) {\bfseries else} \href{axiomas.html#cantBlack}{\tt cant\+Black}(p-\/$>$parent) 
\end{DoxyParagraph}


\begin{DoxyParagraph}{color\+Adecuado}
devuelve un bool que indica si el y el padre, respetan el invariante de coloreo del rb-\/tree

color\+Adecuado\+: puntero(\+Node) $\to$ bool~\newline
 color\+Adecuado(p) $\equiv$ {\bfseries if} p = null {\bfseries then} true {\bfseries else} {\bfseries if} nothing?(p-\/$>$value) $\land$ p-\/$>$color = H\+E\+A\+D\+ER ~\newline
 {\bfseries then} true {\bfseries else} {\bfseries if} p = p-\/$>$parent-\/$>$parent $\land$ p-\/$>$color = black {\bfseries then} true ~\newline
 {\bfseries else} {\bfseries if} (p-\/$>$color = red and p-\/$>$parent-\/$>$color = black) $\lor$ (p-\/$>$color = black and p-\/$>$parent-\/$>$color = black) {\bfseries then} true {\bfseries else} false 
\end{DoxyParagraph}


\begin{DoxyParagraph}{es\+Hoja}
devuelve true si el nodo es hoja

es\+Hoja\+: puntero(\+Node) $\to$ bool~\newline
 es\+Hoja(p) $\equiv$ {\bfseries if} p = null {\bfseries then} false {\bfseries else} {\bfseries if} nothing?(p-\/$>$value) {\bfseries then} false {\bfseries else} {\bfseries if} p-\/$>$child\mbox{[}0\mbox{]} = null $\lor$ p-\/$>$child\mbox{[}1\mbox{]} = null {\bfseries then} true else false 
\end{DoxyParagraph}


\begin{DoxyParagraph}{esta?}
devuele true si el elemento pertenece al (arbol/diccionario)

esta?\+: puntero(\+Node) x puntero(\+Node) $\to$ bool~\newline
 esta?(p1,p2) $\equiv$ $\ast$p1 en elementos(p2) 
\end{DoxyParagraph}


\begin{DoxyParagraph}{sin\+Repetidos}
devuelve true si no hay elementos repetidos en el arbol/diccionario

sin\+Repetidos\+: secu(\+Key) $\to$ bool~\newline
 sin\+Repetidos(sec) $\equiv$ if vacia?(sec) {\bfseries then} true {\bfseries else} {\bfseries if} esta?(prim(sec),fin(sec)) {\bfseries then} false {\bfseries else} sin\+Repetidos(fin(sec)) 
\end{DoxyParagraph}


\begin{DoxyParagraph}{header\+To\+Secu}
dado un puntero a nodo te devuelve una secuencia de Key

header\+To\+Secu\+: puntero(\+Node) $\to$ secu(\+Key)~\newline
 header\+To\+Secu(p) $\equiv$ {\bfseries if} p = null {\bfseries then} $<$ $>$ {\bfseries else} {\bfseries if} nothing?(p-\/$>$value) {\bfseries then} header\+To\+Secu(p-\/$>$parent) ~\newline
 {\bfseries else} $\pi_1$(p-\/$>$value) o header\+To\+Secu(p-\/$>$child\mbox{[}0\mbox{]}) \& header\+To\+Secu(p-\/$>$child\mbox{[}1\mbox{]}) 
\end{DoxyParagraph}


\begin{DoxyParagraph}{arbolK}
devuelve un arbol cantidad de niveles igual a K (arbol de cardinal finito)

arbolK\+: puntero(\+Node) x nat $\to$ A\+B(puntero(\+Nodo))~\newline
 arbol\+K(p) $\equiv$ {\bfseries if} n = 0 {\bfseries then} Bin(nil , p , nil) else bin(arbolK(p-\/$>$child\mbox{[}0\mbox{]} , n-\/1) , p , arbolK(p-\/$>$child\mbox{[}1\mbox{]} , n-\/1)) ~\newline
 arbol\+K(p) $\equiv$ {\bfseries if} p = null {\bfseries then} bin(nil,nil,nil) {\bfseries else} {\bfseries if} nothing?(p-\/$>$value) {\bfseries then} ~\newline
 arbolK(p-\/$>$parent) {\bfseries else} bin(arbolK(p-\/$>$child\mbox{[}0\mbox{]}) , p , arbolK(p-\/$>$child\mbox{[}1\mbox{]})) 
\end{DoxyParagraph}


\begin{DoxyParagraph}{elementos}
devuelve el conjunto de elementos

elementos\+: puntero(\+Node) $\to$ conj(value)~\newline
 elementos(p) $\equiv$ if p = null {\bfseries then} vacio {\bfseries else} {\bfseries if} nothing?(p-\/$>$value) {\bfseries then} elementos(p-\/$>$parent) ~\newline
 {\bfseries else} Ag(p-\/$>$value,vacio) U elementos(n-\/$>$child\mbox{[}0\mbox{]}) U elementos(n-\/$>$child\mbox{[}1\mbox{]}) 
\end{DoxyParagraph}


\begin{DoxyParagraph}{es\+Diccionario?}
Retorna true si la secuencia representa un diccionario

es\+Diccionario?\+: secu(tupla( $\alpha$, $\beta$)) $\to$ bool~\newline
 es\+Diccionario?(s) $\equiv$ sin\+Repetidos?(\href{axiomas.html#primeros}{\tt primeros}(s)) 
\end{DoxyParagraph}


\begin{DoxyParagraph}{primeros}
Proyecta las primeras componentes de una secuencia de pares

primeros\+: secu(tupla( $\alpha$, $\beta$)) $\to$ secu( $\alpha$)~\newline
 primeros(s) $\equiv$ {\bfseries if} vacia?(s) {\bfseries then} $<$$>$ {\bfseries else} $\pi_1$(prim(s)) $\bullet$ \href{axiomas.html#primeros}{\tt primeros}(fin(s)) {\bfseries fi} 
\end{DoxyParagraph}
