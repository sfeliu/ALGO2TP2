En esta pagina, y por conveniencia, se listan todos los axiomas y proposiciones auxiliares requeridos para formalizar los invariantes de representación y las funciones de abstracción. Previamente se presentan los renombres de los tipos usados.

\begin{DoxyParagraph}{Renombres de tipos}

\end{DoxyParagraph}

\begin{DoxyItemize}
\item Node es tupla(child\+: arreglo\mbox{[}2\mbox{]} de puntero(\+Node), parent\+: puntero(\+Node), color\+: Color, value\+: Value)
\item Value es Maybe(value\+\_\+type)
\item value\+\_\+type es tupla(clave\+: Key, significado\+: Meaning)
\end{DoxyItemize}

El T\+AD Maybe( $\alpha$) representa un tipo $\alpha$ extendido con un valor nulo. En otras palabras, el T\+AD Maybe se puede usar para representar los valores de los nodos, donde el nodo cabecera no tiene valor y los nodos internos sí tienen valor. (Esto independientemente de si se implementa con herencia o con un puntero o de otra forma.) Tiene dos observadores\+:
\begin{DoxyItemize}
\item nothing?(x)\+: que indica si x tiene un valor nulo, y
\item data(x)\+: que devuelve el valor de x, suponiendo que no es inválido.
\end{DoxyItemize}

La especificación de este T\+AD queda como ejercicio (no obligatorio).\hypertarget{axiomas_sec-Axiomas}{}\subsection{Axiomas y proposiciones auxiliares}\label{axiomas_sec-Axiomas}
En esta sección se deben incluir todos los axiomas y proposiciones auxiliares que se usen para describir los invariantes de representación, las funciones de abstracción, las precondiciones y las postcondiciones.

\begin{DoxyRemark}{Comentarios}
Recordar incluir un alias en el archivo doxyfile a fin de poder referenciar automaticamente a cada axioma desde las otras páginas.
\end{DoxyRemark}
Se muestran algunos ejemplos a continuación.

\begin{DoxyParagraph}{Arbol\+Value}
A partir de un puntero a nodo construye el arbol binario al que pertenece ese node.

Arbol\+Value\+: puntero(\+Node) $\to$ ab(\+Value) Arbol\+Value(p) $\equiv$ {\bfseries if} p = null {\bfseries then} nil ~\newline
 {\bfseries else} {\bfseries if} nothing?($\ast$p.value) {\bfseries then} bin(\href{axiomas.html#ArbolValue}{\tt Arbol\+Value}($\ast$($\ast$p.parent).child\mbox{[}0\mbox{]} , dato($\ast$($\ast$p.parent).value) , \href{axiomas.html#ArbolValue}{\tt Arbol\+Value}($\ast$($\ast$p.parent).child\mbox{[}1\mbox{]}) )) ~\newline
 {\bfseries else} bin(\href{axiomas.html#ArbolValue}{\tt Arbol\+Value}($\ast$p.child\mbox{[}0\mbox{]} ) , dato($\ast$p.value) , \href{axiomas.html#ArbolValue}{\tt Arbol\+Value}($\ast$p.child\mbox{[}1\mbox{]})) {\bfseries fi} {\bfseries fi} 
\end{DoxyParagraph}


\begin{DoxyParagraph}{dame\+Header}
A partir de un nodo devuelve el dame\+Header.

dame\+Header\+: puntero(\+Node) $\to$ Node~\newline
 dame\+Header(p) $\equiv$ {\bfseries if} nothing (($\ast$p).value) {\bfseries then} $\ast$p {\bfseries else} \href{axiomas.html#dameHeader}{\tt dame\+Header}(($\ast$p).padre) {\bfseries fi} 
\end{DoxyParagraph}


\begin{DoxyParagraph}{en\+Rango}
A partir de cuatro punteros a nodo devuelve true si y solo si el primer y segundo parametro estan acotados inferiormente por el tercero y superiormente por el cuarto.

en\+Rango\+: puntero(\+Node) x puntero(\+Node) x puntero(\+Node) x puntero(\+Node) $\to$ bool~\newline
 en\+Rango(p1,p2,p3,p4) $\equiv$ data($\ast$($\ast$p3.value).clave $\leq$ data($\ast$p2.value).clave $\land$ data($\ast$p4.value).clave $\geq$ data($\ast$p2.value).clave) $\land$ data($\ast$($\ast$p3.value).clave $\leq$ data($\ast$p1.value).clave $\land$ data($\ast$p4.value).clave $\geq$ data($\ast$p1.value).clave) 
\end{DoxyParagraph}


\begin{DoxyParagraph}{anteriores\+De}
Devuelve la sublista ordenada desde el comienzo de la lista hasta v, sin incluir.

anteriores\+De\+: secu(value\+\_\+type) s x value\+\_\+type v $\to$ secu(value\+\_\+type) \{esta?(v,s)\}~\newline
 anteriores\+De(s, v) $\equiv$ {\bfseries if} prim(s) == v {\bfseries then} $<$$>$ {\bfseries else} prim(s) $\bullet$ \href{axiomas.html#anterioresDe}{\tt anteriores\+De}(fin(s),v) {\bfseries fi} 
\end{DoxyParagraph}


\begin{DoxyParagraph}{siguientes\+De}
Devuelve la sublista ordenada desde v, incluyendolo, hasta el final de la lista.

siguientes\+De\+: secu(value\+\_\+type) s x value\+\_\+type v $\to$ secu(value\+\_\+type) \{esta?(v,s)\}~\newline
 siguientes\+De(s, v) $\equiv$ {\bfseries if} prim(s) == v {\bfseries then} v $\bullet$ $<$$>$ {\bfseries else} \href{axiomas.html#siguientesDe}{\tt siguientes\+De}(com(s),v) $\bullet$ ult(s) {\bfseries fi} 
\end{DoxyParagraph}


\begin{DoxyParagraph}{menor\+Lexico}
compara lexicograficamente 2 conjuntos.

menor\+Lexico\+: conj(key) $\times$ conj(key) $\to$ bool ~\newline
 menor\+Lexico(c1 ,c2) $\equiv$ {\bfseries if} $\emptyset$?(c2) {\bfseries then} false ~\newline
 {\bfseries else} {\bfseries if} $\emptyset$?(c1) {\bfseries then} true ~\newline
 {\bfseries else} {\bfseries if} \href{axiomas.html#minimo}{\tt minimo}(c1,dame\+Uno(c1)) = \href{axiomas.html#minimo}{\tt minimo}(c2, dame\+Uno(c2)) {\bfseries then} (c1 $-$ \{\href{axiomas.html#minimo}{\tt minimo}(c1,dame\+Uno(c1))\}, c2 $-$ \{\href{axiomas.html#minimo}{\tt minimo} (c2, dame\+Uno(c2))\} ) ~\newline
 {\bfseries else} \href{axiomas.html#minimo}{\tt minimo}(c1,dame\+Uno(c1)) $<$ \href{axiomas.html#minimo}{\tt minimo}(c2,dame\+Uno(c2)) {\bfseries fi} {\bfseries fi} {\bfseries fi} 
\end{DoxyParagraph}


\begin{DoxyParagraph}{minimo}
devuelve el elemento menor de un conjunto.

minimo\+: conj(key) c $\times$ Key k $\to$ key \{k $\in$ c\} ~\newline
 minimo(c,k) $\equiv$ {\bfseries if} $\emptyset$?(c) {\bfseries then} k ~\newline
 {\bfseries else} {\bfseries if} k $<$ dame\+Uno(c) {\bfseries then} \href{axiomas.html#minimo}{\tt minimo}(sin\+Uno(c),k) ~\newline
 {\bfseries else} \href{axiomas.html#minimo}{\tt minimo}(sin\+Uno(c) ,dame\+Uno(c)) {\bfseries fi} {\bfseries fi} 
\end{DoxyParagraph}


\begin{DoxyParagraph}{cantidad\+De\+Elementos}
devuelve la cantidad de elementos que tiene el diccionario

cantidad\+De\+Elementos\+: puntero(\+Node) $\to$ nat~\newline
 cantidad\+De\+Elementos(p) $\equiv$ {\bfseries if} p = null {\bfseries then} 0 ~\newline
 {\bfseries else} {\bfseries if} nothing?($\ast$p.value) {\bfseries then} \href{axiomas.html#cantidadDeElementos}{\tt cantidad\+De\+Elementos}($\ast$p.parent) ~\newline
 {\bfseries else} 1 + \href{axiomas.html#cantidadDeElementos}{\tt cantidad\+De\+Elementos}($\ast$p.child\mbox{[}0\mbox{]}) + \href{axiomas.html#cantidadDeElementos}{\tt cantidad\+De\+Elementos}($\ast$p.child\mbox{[}1\mbox{]}) {\bfseries fi} {\bfseries fi} 
\end{DoxyParagraph}


\begin{DoxyParagraph}{es\+A\+DB}
devuelve una bool indicando si el arbol tiene una relacion de orden total (cada nodo con sus hijos)

es\+A\+DB\+: puntero(\+Node) $\to$ bool~\newline
 es\+A\+D\+B(p) $\equiv$ ( $\forall$ n\+:Node) (n $\in$ header\+To\+Secu\mbox{[}\&header\mbox{]}) $\Rightarrow_{\rm L}$ ((( $\forall$ n\textquotesingle{}\+:Node) (n\textquotesingle{} $\in$ header\+To\+Secu\mbox{[}0\mbox{]}) $\Rightarrow_{\rm L}$ ($\ast$n\textquotesingle{}.value).clave $<$ ($\ast$n.value).clave) $\land$ (( $\forall$ n\textquotesingle{}\textquotesingle{}\+:Node) (n\textquotesingle{}\textquotesingle{} $\in$ header\+To\+Secu\mbox{[}1\mbox{]}) $\Rightarrow_{\rm L}$ ($\ast$n\textquotesingle{}\textquotesingle{}.value).clave $>$ ($\ast$n.value).clave)) 
\end{DoxyParagraph}


\begin{DoxyRefDesc}{Obsoleto}
\item[\hyperlink{deprecated__deprecated000001}{Obsoleto}]Esta operación (y bastantes otras) son muy dificiles de leer 

Además, si hace lo que creo que hace, no verifica correctamente que sea un A\+DB, piensen en un arbol con la raiz igual a 5, su hijo izq igual a 3 y el hijo derecho de 3 igual a 8, eso no es un A\+BB pero cumple su proposición.\end{DoxyRefDesc}


\begin{DoxyParagraph}{es\+Menor}
devuelve una bool indicando si el valor del primer puntero es menor al del segundo.

es\+Menor\+: puntero(\+Node) $\times$ puntero(\+Node) $\to$ bool~\newline
 es\+Menor(p, p\textquotesingle{}) $\equiv$ data(($\ast$p).value).clave $<$ data(($\ast$p).value).clave 
\end{DoxyParagraph}


\begin{DoxyParagraph}{cant\+Black}
dado un puntero a nodo, devuelve la cantidad de nodos negros hay hasta llegar al root

cant\+Black\+: puntero(\+Node) $\to$ nat~\newline
 cant\+Black(p) $\equiv$ {\bfseries if} p = null {\bfseries then} 0 ~\newline
 {\bfseries else} {\bfseries if} nothing?($\ast$p.value) {\bfseries then} 0 ~\newline
 {\bfseries else} {\bfseries if} p = ($\ast$($\ast$p).parent).parent) {\bfseries then} 1 ~\newline
 {\bfseries else} {\bfseries if} $\ast$p.color = black {\bfseries then} 1 + \href{axiomas.html#cantBlack}{\tt cant\+Black}($\ast$p.parent) ~\newline
 {\bfseries else} \href{axiomas.html#cantBlack}{\tt cant\+Black}($\ast$p.parent) {\bfseries fi} {\bfseries fi} {\bfseries fi} 
\end{DoxyParagraph}


\begin{DoxyParagraph}{color\+Adecuado}
devuelve un bool que indica si el y el padre, respetan el invariante de coloreo del rb-\/tree

color\+Adecuado\+: puntero(\+Node) $\to$ bool~\newline
 color\+Adecuado(p) $\equiv$ {\bfseries if} p = null $\lor_{\rm L}$ (nothing?($\ast$p.value) $\land$ $\ast$p.color = Header) $\lor_{\rm L}$ ~\newline
 \mbox{[}$\ast$p.color = red $\land$ $\ast$($\ast$p.parent).color = black\mbox{]} $\lor$ \mbox{[}$\ast$p.color = black $\land$ ($\ast$($\ast$p.parent).color = black $\land$ $\ast$($\ast$p.parent).color = red)\mbox{]} $\lor_{\rm L}$ ($\ast$p.parent != null $\land_{\rm L}$ ($\ast$p.parent).color = header $\land_{\rm L}$ $\ast$p.color = black ) {\bfseries then} true {\bfseries else} false 
\end{DoxyParagraph}


\begin{DoxyParagraph}{es\+Hoja}
devuelve true si el nodo es hoja

es\+Hoja\+: puntero(\+Node) $\to$ bool~\newline
 es\+Hoja(p) $\equiv$ {\bfseries if} p = null $\lor_{\rm L}$ nothing?($\ast$p.value) {\bfseries then} false {\bfseries else} $\ast$p.child\mbox{[}0\mbox{]} = null $\lor$ $\ast$p.child\mbox{[}1\mbox{]} = null {\bfseries fi} 
\end{DoxyParagraph}


\begin{DoxyParagraph}{esta\+Ptr?}
devuele true si el elemento pertenece al (arbol/diccionario)

esta\+Ptr?\+: puntero(\+Node) x puntero(\+Node) $\to$ bool~\newline
 esta\+Ptr?(p1,p2) $\equiv$ $\ast$p1.value $\in$ \href{axiomas.html#elementos}{\tt elementos}(p2) 
\end{DoxyParagraph}


\begin{DoxyParagraph}{sin\+Repetidos}
devuelve true si no hay elementos repetidos en el arbol/diccionario

sin\+Repetidos\+: secu(\+Key) $\to$ bool~\newline
 sin\+Repetidos(sec) $\equiv$ if vacia?(sec) {\bfseries then} true ~\newline
 {\bfseries else} {\bfseries if} esta?(prim(sec),fin(sec)) {\bfseries then} false ~\newline
 {\bfseries else} \href{axiomas.html#sinRepetidos}{\tt sin\+Repetidos}(fin(sec)) {\bfseries fi} 
\end{DoxyParagraph}


\begin{DoxyParagraph}{header\+To\+Secu}
dado un puntero a nodo te devuelve una secuencia de Key

header\+To\+Secu\+: puntero(\+Node) $\to$ secu(\+Key)~\newline
 header\+To\+Secu(p) $\equiv$ {\bfseries if} p = null {\bfseries then} $<$ $>$ ~\newline
 {\bfseries else} {\bfseries if} nothing?($\ast$p.value) {\bfseries then} \href{axiomas.html#headerToSecu}{\tt header\+To\+Secu}($\ast$p.parent) ~\newline
 {\bfseries else} data($\ast$p.value).clave $\bullet$ \href{axiomas.html#headerToSecu}{\tt header\+To\+Secu}($\ast$p.child\mbox{[}0\mbox{]}) \& \href{axiomas.html#headerToSecu}{\tt header\+To\+Secu}($\ast$p.child\mbox{[}1\mbox{]}) {\bfseries fi} {\bfseries fi} 
\end{DoxyParagraph}


\begin{DoxyParagraph}{arbolK}
devuelve un arbol cantidad de niveles igual a K (arbol de cardinal finito)

arbolK\+: puntero(\+Node) x nat $\to$ A\+B(puntero(\+Nodo))~\newline
 arbol\+K(p) $\equiv$ {\bfseries if} p = null $\lor$ p = null {\bfseries then} nil ~\newline
 {\bfseries else} {\bfseries if} nothing?($\ast$p.value) {\bfseries then} arbolK(p-\/$>$parent , n) ~\newline
 {\bfseries else} AB(arbolK($\ast$p.child , n-\/1), p ,arbolK($\ast$p.child\mbox{[}1\mbox{]} , n-\/1)) {\bfseries fi} {\bfseries fi} 
\end{DoxyParagraph}


\begin{DoxyParagraph}{elementos}
devuelve el conjunto de elementos

elementos\+: puntero(\+Node) $\to$ conj(value)~\newline
 elementos(p) $\equiv$ if p = null {\bfseries then} vacio ~\newline
 {\bfseries else} {\bfseries if} nothing?($\ast$p.value) {\bfseries then} \href{axiomas.html#elementos}{\tt elementos}($\ast$p.parent) ~\newline
 {\bfseries else} Ag($\ast$p.value,vacio) U \href{axiomas.html#elementos}{\tt elementos}($\ast$p.child\mbox{[}0\mbox{]}) U \href{axiomas.html#elementos}{\tt elementos}($\ast$p.child\mbox{[}1\mbox{]}) {\bfseries fi} 
\end{DoxyParagraph}


\begin{DoxyParagraph}{hijos\+Header}
Retorna true si esta vacio y los hijos del header es el mismo o si los hijos pertenecen.

hijos\+Header\+: Node $\to$ bool~\newline
 hijos\+Header(n) $\equiv$ {\bfseries if} n.\+parent = N\+U\+LL {\bfseries then} n.\+child\mbox{[}0\mbox{]} = \&n $\land$ n.\+child\mbox{[}1\mbox{]} = \&n ~\newline
 {\bfseries else} \href{axiomas.html#estaPtr}{\tt esta\+Ptr}?(n.\+child\mbox{[}0\mbox{]}, \&n) $\land$ \href{axiomas.html#estaPtr}{\tt esta\+Ptr}?(n.\+child\mbox{[}1\mbox{]}, \&n) 
\end{DoxyParagraph}


\begin{DoxyParagraph}{es\+Diccionario?}
Retorna true si la secuencia representa un diccionario

es\+Diccionario?\+: secu(tupla( $\alpha$, $\beta$)) $\to$ bool~\newline
 es\+Diccionario?(s) $\equiv$ \href{axiomas.html#sinRepetidos}{\tt sin\+Repetidos}?(\href{axiomas.html#primeros}{\tt primeros}(s)) 
\end{DoxyParagraph}


\begin{DoxyParagraph}{son\+Validos}
Retorna true si los nodos tienen significado valido

son\+Validos\+: Puntero(\+Node) x Puntero(\+Node) $\to$ bool~\newline
 son\+Validos(p,p\textquotesingle{}) $\equiv$ $\lnot$ (nothing?($\ast$p.value)) $\land$ $\lnot$ (nothing?($\ast$p\textquotesingle{}.value)) 
\end{DoxyParagraph}


\begin{DoxyParagraph}{estan}
Retorna true si los nodos tienen significado valido

estan\+: Puntero(\+Node) x Puntero(\+Node) $\to$ bool~\newline
 estan(p,p\textquotesingle{}) $\equiv$ \href{axiomas.html#estaPtr}{\tt esta\+Ptr}(p) $\land$ \href{axiomas.html#estaPtr}{\tt esta\+Ptr}(p\textquotesingle{}) 
\end{DoxyParagraph}


\begin{DoxyParagraph}{colores\+Adecuados}
Retorna true si los nodos tienen significado valido

colores\+Adecuados\+: Puntero(\+Node) x Puntero(\+Node) $\to$ bool~\newline
 colores\+Adecuados(s) $\equiv$ \href{axiomas.html#colorAdecuado}{\tt color\+Adecuado}(p) $\land$ \href{axiomas.html#colorAdecuado}{\tt color\+Adecuado}(p\textquotesingle{}) 
\end{DoxyParagraph}


\begin{DoxyParagraph}{primeros}
Proyecta las primeras componentes de una secuencia de pares

primeros\+: secu(tupla( $\alpha$, $\beta$)) $\to$ secu( $\alpha$)~\newline
 primeros(s) $\equiv$ {\bfseries if} vacia?(s) {\bfseries then} $<$$>$ {\bfseries else} $\pi_1$(prim(s)) $\bullet$ \href{axiomas.html#primeros}{\tt primeros}(fin(s)) {\bfseries fi} 
\end{DoxyParagraph}


\begin{DoxyParagraph}{son\+Hoja}
Retorna true si los nodos tienen significado valido

son\+Hoja\+: Puntero(\+Node) x Puntero(\+Node) $\to$ bool~\newline
 son\+Hoja(p,p\textquotesingle{}) $\equiv$ \href{axiomas.html#esHoja}{\tt es\+Hoja}(p) $\land$ \href{axiomas.html#esHoja}{\tt es\+Hoja}(p\textquotesingle{}) 
\end{DoxyParagraph}


\begin{DoxyParagraph}{es\+R\+B\+Tree}
Retorna True si a partir del nodo dado se puede reconstruir un Red-\/\+Black Tree.

es\+R\+B\+Tree\+: Node $\to$ Bool~\newline
 \href{axiomas.html#esRBTree}{\tt es\+R\+B\+Tree}(n) $\equiv$ ( $\exists$ k\+: nat)(\href{axiomas.html#arbolK}{\tt arbolK}(n.\+parent,k) = \href{axiomas.html#arbolK}{\tt arbolK}(n.\+parent,k+1)) $\land_{\rm L}$ ~\newline
 \href{axiomas.html#sinRepetidos}{\tt sin\+Repetidos}(header\+To\+Secu(n.\+parent)) $\land$ \href{axiomas.html#esADB}{\tt es\+A\+DB}(n.\+parent) $\land$ hijos\+Header(header) $\land$ ~\newline
 ( $\forall$ p,p\textquotesingle{}\+:puntero(\+Node))(\href{axiomas.html#sonValidos}{\tt son\+Validos}(p,p\textquotesingle{}) $\land_{\rm L}$ \href{axiomas.html#estan}{\tt estan}(p,p\textquotesingle{},header.\+parent) $\Rightarrow_{\rm L}$ \href{axiomas.html#coloresAdecuados}{\tt colores\+Adecuados}(p,p\textquotesingle{}) $\land$ ~\newline
 ((\href{axiomas.html#sonHoja}{\tt son\+Hoja}(p,p\textquotesingle{})) $\Rightarrow_{\rm L}$ \href{axiomas.html#cantBlack}{\tt cant\+Black}(p)=\href{axiomas.html#cantBlack}{\tt cant\+Black}(p\textquotesingle{})) $\land$ en\+Rango(p,p\textquotesingle{},header.\+child\mbox{[}0\mbox{]},header.\+child\mbox{[}1\mbox{]})) 
\end{DoxyParagraph}


\begin{DoxyRefDesc}{Obsoleto}
\item[\hyperlink{deprecated__deprecated000002}{Obsoleto}]sin\+Repetidos(header\+To\+Secu(n.\+parent)) esto no te soluciona uno de los problemas hablado por Soulignac. Piensen como podrían reescribir a partir del rep de map V\+I\+S\+TO\end{DoxyRefDesc}


\begin{DoxyParagraph}{padreK}
Retorna el k-\/esimo padre de un puntero a nodo.

padreK\+: puntero(\+Node) $\to$ Node~\newline
 padre\+K(p, k) $\equiv$ {\bfseries if} k = 0 $\lor$ ($\ast$p).parent = null {\bfseries then} $\ast$p {\bfseries else} \href{axiomas.html#padreK}{\tt padreK}(($\ast$p).parent, k-\/1) {\bfseries fi} {\bfseries fi} 
\end{DoxyParagraph}


\begin{DoxyParagraph}{elementos\+MayoresA}
Retorna la secuencia ordenada de elementos mayores al pasado como parametro.

elementos\+MayoresA\+: deicc(value) x conj(key) x key $\to$ secu(value)~\newline
 elementos\+Mayores\+A(d, c, k) $\equiv$ {\bfseries if} $\emptyset$(c)? {\bfseries then} $<$ $>$ {\bfseries else} {\bfseries if} maximo(c) $\leq$ k {\bfseries then} \href{axiomas.html#elementosMayoresA}{\tt elementos\+MayoresA}(d, c -\/ maximo(c), k) o $<$maximo(c), obtener(minimo(c), d)$>$ {\bfseries else} \href{axiomas.html#elementosMayoresA}{\tt elementos\+MayoresA}(d, c -\/ maximo(c), k) {\bfseries fi} {\bfseries fi} 
\end{DoxyParagraph}


\begin{DoxyParagraph}{elementos\+MenoresA}
Retorna la secuencia ordenada de elementos menores al pasado como parametro.

elementos\+MenoresA\+: dicc(value) x conj(key) x key $\to$ secu(value)~\newline
 elementos\+Menores\+A(d, c, k) $\equiv$ {\bfseries if} $\emptyset$(c)? {\bfseries then} $<$ $>$ {\bfseries else} {\bfseries if} minimo(c) $<$ k {\bfseries then} $<$minimo(c), obtener(minimo(c), d)$>$ $\bullet$ \href{axiomas.html#elementosMenoresA}{\tt elementos\+MenoresA}(d, c -\/ minimo(c), k) {\bfseries else} \href{axiomas.html#elementosMenoresA}{\tt elementos\+MenoresA}(d, c -\/ minimo(c), k) {\bfseries fi} {\bfseries fi} 
\end{DoxyParagraph}
